\documentclass[11pt]{article}
\usepackage[czech]{babel}
\usepackage[utf8]{inputenc}
\usepackage{graphicx}
\usepackage{a4wide}
\usepackage[square, numbers]{natbib}
\usepackage{algpseudocode}
\usepackage{boxedminipage}
\usepackage{amsthm}
\usepackage{amsmath}
\usepackage{amsfonts}
\usepackage{amssymb}
\usepackage{tikz}
\usepackage{multirow}

%\renewcommand*\rmdefault{iwona}

\paperwidth=210 true mm
\paperheight=297 true mm
\pdfpagewidth=210 true mm
\pdfpageheight=297 true mm

\author{Daniel Meister}
\title{Konstrukce BVH pomocí SAH}


\begin{document}

\begin{center}
\textsc{\LARGE Konstrukce BVH pomocí SAH}\\[0.4cm]
\textsc{\Large Semestrální projekt z A4M39DPG}\\[0.4cm]
\textsc{\large ČVUT, Fakulta elektrotechnická}\\[0.4cm]
\textsc{\normalsize Daniel Meister}\\[0.5cm]
\end{center}

\section*{Zadání}
Naimplementujte stavbu hierarchie obálek podle článku \emph{On fast Construction of SAH based Bounding Volume Hierarchies} \cite{wald}.

\section{Úvod}
\emph{Ray tracing} je pevně spaj s akceleračními strukturami. Jednou z nejvíce používaných akceleračních struktur je hierarchie obálek. S narůstajícím výpočetním výkonem začínáme být schopni \emph{ray tracovat} scény v reálném čase, a proto \emph{ray tracing} začíná pomalu pronikat i do dynamických aplikací jako jsou počítačové hry.

V dynamických scénách se geometrie mění v každém snímku a proto je nutné akcelerační strukturu v každém snímku aktualizovat. Přestavba akcelerační struktury je nejpřirozenější způsob aktualizace. Rychlá stavba heirarchie obálek je tedy velmi žádoucí.

\section{Hierarchie obálek (BVH)}
Hierarchie obálek je binární strom. Primitiva jsou obsaženy v listech. Každý vnitřní uzel má obálku, která obsahuje všechny její potomky. Kořen obsahuje obálku celé scény.

Nejnákladnější operací v \emph{ray tracingu} je nalezení nejbližšího průsečíku pro daný paprsek. BVH může snížit časovou složitost této operace až na $\mathcal O(\log n)$, kde $n$ počet pritiv ve scéně.

\subsection{Stavba BVH}
BVH lze stavět shora dolů (\emph{top-down}), nebo odspoda nahoru (\emph{bottom-up}). Stavbou odspoda nahoru se zde nebudu zabývat. Stavba shora dolů probíhá tak, že se primitiva aktuálního uzlu rozdělí na dvě disjunktní části. Pro každou část se spočítá nová obálka. Každá část se stejným způsobem rekurzivně dále dělí. Dělení se se zastaví až nastane některá z ukončovacích podmínek.

\subsection{Surface Area Heuristic (SAH)}
Pokud aktuální uzel obsahuje $n$ primitiv, pak ho můžeme rozdělit $\sum_{i = 1}^{n - 1} {n\choose i}$ způsoby, což je exponenciálně mnoho. SAH je \emph{greedy} heuristika, která se uplatňuje při dělení primitiv. SAH pracuje s osově zarovnanými kvádry. SAH se snaží oddělit středy primitiv rovinou takovou, že jedna ze souřadných os je na ní kolmá. Tímto způsobem můžeme rozdělit primitiva $n - 1$ způsoby pro každou ze souřadných os. Cenu dělící roviny $\rho$ lze vyjádřit pomocí vztahu (\ref{cost}), kde $c_t$ je cena traverzace uzlu, $c_i$ cena výpočtu nejbližšího průsečíku,  $S$ je povrch původní obálky, $S_l$ (resp. $S_r$) je povrch levé (resp. pravé) obálky a $n_l$ (resp. $n_r$) je počet primitiv v levé (resp. v pravé) obálce. Po zanedbání konstantních faktorů lze vztah (\ref{cost}) zjednodušit na vztah (\ref{costsimple}). Chtěli bychom nejlevnější dělící rovinu.

\begin{equation}
c_\rho = c_t + c_i\left(\frac{S_l}{S}n_l +\frac{S_r}{S}n_r\right)
\label{cost}
\end{equation}

\begin{equation}
c_\rho = S_l n_l +S_r n_r
\label{costsimple}
\end{equation}

\subsection{Binning}
Binning je jedna z algoritmů, který lze využít k výběru dělící roviny v SAH. Vybere se osa podle, které se bude dělení provádět. Podél této osy se obálka rozdělí na $k$ stejně velkých úseků tzv. binů. Primitiva se roztřídí do jednotlivých binů podle svých středů. Index binu, do kterého se primitivum promítne, lze určit podle vztahu (\ref{proj}), kde $j$ je dělící osa, $b_{min,j}$ a $b_{max,j}$ jsou $j$-té souřadnice parametrů obálky aktuálního uzlu, $c_{i,j}$ je $j$-tá souřadnice středu $i$-tého primitiva. Výraz $1 - \varepsilon$ nám zaručí, že středy primitiv na pravém okraji obálky se promítnou do posledního binu. U každého binu si udržujeme počet promítnutých primitiv a jejich obálku.

\begin{equation}
BinID = \left\lfloor{\frac{ k (1 - \varepsilon) (b_{max, j} - c_{i, j})}{b_{max, j} - b_{min, j}}}\right\rfloor
\label{proj}
\end{equation}

Zde máme $k - 1$ možných dělících rovin, to už je přijatelný počet. Spočítáme jejich ceny a vybereme tu nejlevnější. To provedeme třemi lineárními průchody. V prvním průchodu spočítáme inkluzivní prefixový součet obálek a počtů primitiv zleva přes všechny biny. V druhém průchodu spočítáme exkluzivní prefixový součet obálek a počtů primitiv přes všechny biny. Nyní znamé počty primitiv na levých a pravých stranách pro všech $k - 1$ rovin. V posledním průchodu postupně spočítáme ceny všech rovin a vybereme tu nejlevnější. Z prefixových součtů také známe obálky levého a pravého syna pro každou testovanou dělící rovinu. Příklad prefixových součtů je možné vidět v tabulce (\ref{binning}).

\begin{table}[h]
\begin{center}
\begin{tabular}{|l|c|c|c|c|c|}
\hline
\textbf{Počet primitiv v binech} & 2 & 0 & 4 & 10 & 1\\
\hline
\textbf{První prefixový součet} & 2 & 2 & 6 & 16 & 17\\
\hline
\textbf{Druhý prefixový součet} & 15 & 15  & 11 & 1 & 0\\
\hline
\end{tabular}
\end{center}
\caption{Příklad prefixových součtů.}
\label{binning}
\end{table}

\subsection{Paralelní stavba BVH}
Paralelní stavba BVH je založena na myšlence paralelního zpracování několika podstromů současně. Na začátku máme pouze jeden (pod)strom a to samotný kořen. V článku \cite{wald} jsou popsány dva paralelní algoritmy stavby BVH, které jsou založeny právě na této myšlence.

\subsubsection{Hybridní zpracování}
Zde nejdříve probíhá tzv. horizontální zpracování. V rámci dělelní jednoho uzlu běží $t$ vláken. Každé vlákno má svou vlastní sadu binů. Primitiva zpracovávaného uzlu se rozdělí na $t$ stejně velkých částí a každá část se přiřadí jednomu vláknu. Každé vlákno roztřídí svá primitiva do svých binů a spočítá oba prefixové součty. Poté první vlákno spočítá nejlevnější dělící rovinu. Počty primitiv na levých a pravých stranách získá součtem levých a pravých stran přes všechny sady binů.

Primitiva scény bývaji uložena v nějakém souvislém poli a primitiva aktuálního uzlu jsou uložena v nějakým úsekem tohoto pole. Při dělelní uzlu je potřeba daný úsek přetřídit na levou a pravou část. Toto lze také paralelizovat. Je nutné spočítat tzv. offsety. To jsou indexy, které říkají jednotlivým vláknům, kam mají svá primitiva v poli trojúhelníků umisťovat. Každé vlákno bude mít levý a pravý offset. Levé offsety lze spočítat pomocí exkluzivního prefixového součtu levých stran přes všechna vlákna. A pravé offsety lze spočítat exkluzivním prefixovým součtem pravých stran přes všechna vlákna a navíc se ještě ke každému pravému offsetu přičte počet primitiv na levé straně.

Po dosažení určitého počtu (řádově $10^4$) primitiv v daném uzlu se horizontální zpracování přepne na vertikální zpracování. To znamená, že se vytvoří nové vlákno a to začne sekvenčně zpracovávat daný uzel jako kdyby byl kořen. Horizontální vlákna se vrátí o úroveň výš a pokračují dále.

\subsubsection{Zpracování založené na mřížce (\emph{grid based binning})}
Vstupem tohoto algoritmu je dělící vektor $s = [s_x, s_y, s_z]$. Kořenová obálka se uniformně rozdělí $s_x \times s_y \times s_z$ buněk. Do těchto buněk se podle svých středů roztřídí primitiva scény a ke každé buňce se spočítá její obálka. Poté se každá buňka zpracovává paralelně jako samostatný podstrom. Nevznikne nám tedy jen jeden BVH strom, ale celý BVH les. Nevýhodou této metody je, že při hledání průsečíku se scénou musíme traverzovat každý strom zvlášť. Výhodou je, že zde není potřeba žádná synchronizace.

\begin{figure}[h]
\setlength\fboxsep{1pt}
\centerline{\fbox{\scalebox{.50}{\includegraphics{pic/grid.pdf}}}}
\caption{Rozdělení kořenové buňky pro vektor dělení $s = [4, 2, 2]$.}
\label{grid}
\end{figure}

\section{Popis aplikace}
Aplikaci lze rozdělit na tři části - parsery, BVH a renderer. Pomocí parserů se scéna načte a nastaví. Jsou zde dva parsery. První je jednoduchý \emph{obj} parser, který načítá geometrii scény a materiály. Umí tedy parsovat také \emph{mtl} soubory a načítat difúzní textury ve formátu \emph{png}. K načítání \emph{png} obrázků byla využita knihovna \emph{lodePNG}. Druhý parser načítá soubory s příponou \emph{conf}. Jedná se o textové soubory, které obsahují nastavení kamery a pozice světel. Parametry a jejich hodnoty je možné nalézt v tabulce (\ref{conf}).

\begin{table}[h]
\begin{center}
\begin{tabular}{|c|c|c|}
\hline
\textbf{parametr} & \textbf{hodnota} & \textbf{význam}\\
\hline
from & x y z & pozice kamery\\
\hline
at & x y z & bod určující směr pohledu\\
\hline
up & x y z & up vektor\\
\hline
angle & fovy & field of view\\
\hline
light & x y z & pozice světla\\
\hline
resolution & width height & rozlišení\\
\hline
\end{tabular}
\end{center}
\caption{Parametry a jejich hodnoty v \emph{conf} souboru.}
\label{conf}
\end{table}

Po načtení scény se nad ní podle zvoleného algoritmu vytvoří BVH. Nakonec se scéna vyrenderuje pomocí jednoduchého nerekurzivního \emph{ray traceru}. K vytvoření okna a zobrazení vyrenderovaného obrázku jsem použil knihovnu \emph{GLUT}.

Dále jsem využil třídu \verb!Environment! z knihovny \emph{minimax}. Tato třída mi umožňuje nastavovat další parametry aplikace. Parametry a jejich hodnoty se zapisují do souboru s příponou \emph{env}. Tento soubor je možné zadat aplikaci jako argument z příkazové řádky. Pokud není zadán žádný soubor, pak se použije defaultní nastavení.

Parametry jsou uzavřeny do bloků. Je možné použít tři bloky - \verb!Bvh!, \verb!RayTracer! a \verb!Scene!. Každý blok má své specifické parametry, které je možné nalézt v tabulce (\ref{env}).

\begin{table}[h]
\begin{center}
\begin{tabular}{|c|c|c|c|}
\hline
\textbf{blok} & \textbf{parametr} & \textbf{hodnota} & \textbf{význam}\\
\hline
Bvh & algorithm & sequential, grid, hybrid & BVH algoritmus\\
\hline
Bvh & gridResolution & x y z & rozlišení mřížky\\
\hline
Bvh & numberOfWorkers & n & počet horizontálních vláken\\
\hline
Bvh & hybridThreshold  & n & práh přepnutí na vertikální zpracování\\
\hline
Bvh & numberOfBins & n & počet binů\\
\hline
Bvh & maxTriangles & n & maximální počet trojúhelníků v listech\\
\hline
Scene & filename & path & cesta k \emph{obj} souboru\\
\hline
Scene & config & path & cesta k \emph{conf} souboru\\
\hline
RayTracer & shading & flat nebo smooth & stínování\\
\hline
RayTracer & textures & true nebo false & zapnutí/vypnutí texturování\\
\hline
\end{tabular}
\end{center}
\caption{Parametry a jejich hodnoty v \emph{env} souboru.}
\label{env}
\end{table}

\section{Naměřené výsledky}
Testování probíhalo na sestavě:
\begin{itemize}
\item CPU: Intel Core2 Quad, Q9650, 3.00 GHz, 12 MB cache
\item paměť: 8 GB RAM, DDR3
\item OS: Windows 7, 64 bit
\item překladač: MS Visual Studio 17.00
\end{itemize}

\noindent
Všechny scény jsem testoval s rozlišením $800 \times 800$. Každá scéna obsahuje jen jedno světlo, proto na každý pixel připadá jeden primární paprsek a jeden stínový paprsek. Výsledné obrázky je možné nalézt v příloze (\ref{pics}). 

Při výstavbě BVH se využívá šestnáct binů, což je doporučený počet v článku \cite{wald}. V tabulkách je vždy uveden testovaný model s počtem trojúhelníků, průměrný počet traverzovaných uzlů na jeden paprsek, průměrný počet incidenčních operací na jeden paprsek, cena BVH stromu, čas stavby BVH jednotlivých variant a čas vykreslení scény. Cena BVH stromu je vyjádřena pomocí rekurzivního SAH vztahu s koeficienty $c_t = 3$ a $c_i = 2$.

\begin{table}[h]
\begin{center}
\begin{tabular}{|c|c|c|c|c|c|c|c|c|}
\hline
model & počet $\Delta$ & trav. uzly & testované $\Delta$ & $cost_{BVH}$ & $t_{BVH_{seq}}$ & $t_{BVH_{hybrid}}$ & $t_{render}$\\
\hline
A10 & 218652 & 4 & 1 & 93 & 480 ms & 380 ms & 1,9 s\\
\hline
armadillo & 345944 & 10 & 2 & 87 & 762 ms & 579 ms & 2,8 s\\
\hline
city & 68497 & 153 & 68 & 267 & 129 ms & 377 ms & 18,9 s\\
\hline
city 2 & 75420 & 5 & 3 & 118 & 130 ms & 416 ms & 2,1 s\\
\hline
conference & 282755 & 229 & 67 & 616 & 624 ms & 570 ms & 27,8 s\\
\hline
fairy forest & 173577 & 29 & 7 & 74 & 380 ms & 319 ms & 5,8 s\\
\hline
park & 29174 & 30 & 11 & 94 & 83 ms & 123 ms & 5,6 s\\
\hline
sibenik & 80479 & 167 & 43 & - & 165 ms & 510 ms & 20,5 s\\
\hline
teapots & 200748 & 143 & 40 & 259 & 408 ms & 349 ms & 17,8 s\\
\hline
\end{tabular}
\end{center}
\caption{Sekvenční a hybridní varianta s max. 4 trojúhelníky v listech.}
\label{test4seq}
\end{table}

\begin{table}[h]
\begin{center}
\begin{tabular}{|c|c|c|c|c|c|c|c|c|}
\hline
model & počet $\Delta$ & trav. uzly & testované $\Delta$ & $cost_{BVH}$ & $t_{BVH_{seq}}$ & $t_{BVH_{grid}}$ & $t_{render}$\\
\hline
A10 & 218652 & 8 & 1 & 84 & 480 ms & 355 ms & 2,2 s\\
\hline
armadillo & 345944 & 12 & 2 & 81 & 762 ms & 533 ms & 3,0 s\\
\hline
city & 68497 & 143 & 59 & 210 & 129 ms & 119 ms & 17,4 s\\
\hline
city 2 & 75420 & 8 & 2 & 131 & 130 ms & 126 ms & 2,3 s\\
\hline
conference & 282755 & 140 & 50 & 488 & 624 ms & 477 ms & 17,7 s\\
\hline
fairy forest & 173577 & 33 & 8 & 69 & 380 ms & 286 ms & 6,0 s\\
\hline
park & 29174 & 28 & 8 & 68 & 83 ms & 54 ms & 5,0 s\\
\hline
sibenik & 80479 & 197 & 53 & - & 165 ms & 140 ms & 24,6 s\\
\hline
teapots & 200748 & 70 & 24 & 209 & 408 ms & 324 ms & 9,1 s\\
\hline
\end{tabular}
\end{center}
\caption{Varianta s mřížkou $2 \times 2 \times 2$ s max. 4 trojúhelníky v listech.}
\label{test4grid}
\end{table}

\begin{table}[h]
\begin{center}
\begin{tabular}{|c|c|c|c|c|c|c|c|c|}
\hline
model & počet $\Delta$ & trav. uzly & testované $\Delta$ & $cost_{BVH}$ & $t_{BVH_{seq}}$ & $t_{BVH_{hybrid}}$ & $t_{render}$\\
\hline
A10 & 218652 & 3 & 6 & 153 & 241 ms & 152 ms & 2,0 s\\
\hline
armadillo & 345944 & 8 & 14 & 142 & 432 ms & 256 ms & 3,1 s\\
\hline
city & 68497 & 98 & 437 & 491 & 55 ms & 341 ms & 26,7 s\\
\hline
city 2 & 75420 & 2 & 8 & 132 & 61 ms & 370 ms & 2,1 s\\
\hline
conference & 282755 & 165 & 508 & 1075 & 331 ms & 202 ms & 37,2 s\\
\hline
fairy forest & 173577 & 24 & 40 & 120 & 188 ms & 125 ms & 6,3 s\\
\hline
park & 29174 & 20 & 65 & 159 & 54 ms & 89 ms & 6,5 s\\
\hline
sibenik & 80479 & 131 & 347 & 490 & 71 ms & 391 ms & 27,2 s\\
\hline
teapots & 200748 & 108 & 293 & 422 & 193 ms & 168 ms & 24,2 s\\
\hline
\end{tabular}
\end{center}
\caption{Sekvenční a hybridní varianta s max. 32 trojúhelníky v listech.}
\label{test32seq}
\end{table}

\begin{table}[h]
\begin{center}
\begin{tabular}{|c|c|c|c|c|c|c|c|c|}
\hline
model & počet $\Delta$ & trav. uzly & testované $\Delta$ & $cost_{BVH}$ & $t_{BVH_{seq}}$ & $t_{BVH_{grid}}$ & $t_{render}$\\
\hline
A10 & 218652 & 6 & 8 & 139 & 241 ms & 136 ms & 2,3 s\\
\hline
armadillo & 345944 & 11 & 13 & 136 & 432 ms & 196 ms & 3,2 s\\
\hline
city & 68497 & 92 & 380 & 389 & 55 ms & 37 ms & 24,2 s\\
\hline
city 2 & 75420 & 6 & 8 & 168 & 61 ms & 36 ms & 2,2 s\\
\hline
conference & 282755 & 97 & 316 & 842 & 331 ms & 155 ms & 22,8 s\\
\hline
fairy forest & 173577 & 26 & 98 & 145 & 188 ms & 100 ms & 8,0 s\\
\hline
park & 29174 & 21 & 46 & 121 & 54 ms & 23 ms & 5,7 s\\
\hline
sibenik & 80479 & 149 & 404 & 433 & 71 ms & 43 ms & 31,6 s\\
\hline
teapots & 200748 & 56 & 127 & 322 & 193 ms & 89 ms & 11,8 s\\
\hline
\end{tabular}
\end{center}
\caption{Varianta s mřížkou $2 \times 2 \times 2$ s max. 32 trojúhelníky v listech.}
\label{test32grid}
\end{table}

V tabulkách (\ref{test4seq}, \ref{test32seq}) je možné vidět výsledky testování hybridní varianty s maximálně 4 a 32 trojúhelníky v listech. Můžete zde nalézt časové srovnání se sekvenční variantou. Práh přepnutí z horizontálního zpracování na vertikální byl globálně nastaven na 20000 trojúhelníků. K horizontálnímu zpracování jsem použil 4 vlákna. Z výsledků je možné usoudit, že hybridní zpracování se vyplatí pro větší scény, které obsahují alespoň 150000 trojúhelníků. Pro malé scény je vhodnější použít sekvenční variantu.

V tabulkách (\ref{test4grid}, \ref{test32grid}) je možné vidět výsledky testování varianty s mřížkou s maximálně 4 a 32 trojúhelníky v listech. Můžete zde vidět srovnání se sekvenčním řešením. Velikost mřížky byla globálně nastavena na $2 \times 2 \times 2$ buněk. Z výsledků vyplývá, že stavba BVH založená na mřížce je ve všech případech rychlejší než hybridní varianta. Také zvládá lépe menší scény. 

Překvapivé je, že renderovací čas u varianty s mřížkou je ve většině testovaných případů lepší než renderovací čas hybridní varianty.

\section{Závěr}
V úvodních sekcích byla vysvětlena teorie nutné k realizaci zadaného projektu. Poté byla aplikaci úspěšně implementována. Byly implementovány oba paralelní algoritmy z článku \cite{wald}. V závěrečné sekci byla aplikace řádně otestována a zhodnocena. Lze konstatovat, že zadání bylo úspněšně splněno.

\vspace{3.0cm}

\bibliographystyle{csplainnat}
\bibliography{reference}

\appendix

\newpage
\section{Obrázky}
\label{pics}

\begin{figure}[h]
\setlength\fboxsep{1pt}
\centerline{\fbox{\scalebox{.3}{\includegraphics{pic/a10.png}}}}
\caption{Scéna \emph{A10}}
\end{figure}

\begin{figure}[h]
\setlength\fboxsep{1pt}
\centerline{\fbox{\scalebox{.3}{\includegraphics{pic/armadillo.png}}}}
\caption{Scéna \emph{armadillo}}
\end{figure}

\begin{figure}[h]
\setlength\fboxsep{1pt}
\centerline{\fbox{\scalebox{.3}{\includegraphics{pic/city.png}}}}
\caption{Scéna \emph{city}}
\end{figure}

\begin{figure}[h]
\setlength\fboxsep{1pt}
\centerline{\fbox{\scalebox{.3}{\includegraphics{pic/city2.png}}}}
\caption{Scéna \emph{city 2}}
\end{figure}

\begin{figure}[h]
\setlength\fboxsep{1pt}
\centerline{\fbox{\scalebox{.3}{\includegraphics{pic/conference.png}}}}
\caption{Scéna \emph{conference}}
\end{figure}

\begin{figure}[h]
\setlength\fboxsep{1pt}
\centerline{\fbox{\scalebox{.3}{\includegraphics{pic/fairy_forest.png}}}}
\caption{Scéna \emph{fairy forest}}
\end{figure}

\begin{figure}[h]
\setlength\fboxsep{1pt}
\centerline{\fbox{\scalebox{.3}{\includegraphics{pic/park.png}}}}
\caption{Scéna \emph{park}}
\end{figure}

\begin{figure}[h]
\setlength\fboxsep{1pt}
\centerline{\fbox{\scalebox{.3}{\includegraphics{pic/sibenik.png}}}}
\caption{Scéna \emph{sibenik}}
\end{figure}

\begin{figure}[h]
\setlength\fboxsep{1pt}
\centerline{\fbox{\scalebox{.3}{\includegraphics{pic/teapots.png}}}}
\caption{Scéna \emph{teapots}}
\end{figure}

\begin{figure}[h]
\setlength\fboxsep{1pt}
\centerline{\fbox{\scalebox{.3}{\includegraphics{pic/fairy_forest_tex.png}}}}
\caption{Scéna \emph{fairy forest} s texturami}
\end{figure}

\end{document}

